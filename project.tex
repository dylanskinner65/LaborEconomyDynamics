\documentclass[11pt]{amsart}
%%% WARNING: Do NOT change the page size, fonts, or margins!  Penalties will apply.

\usepackage{placeins} %enables \FloatBarrier, that prevents floats from going below it.
\usepackage{amsmath, amssymb, amsfonts, amsthm, mathtools, bm}
\usepackage{color,comment}
\usepackage{geometry}
\usepackage{hyperref}
\usepackage[english]{babel}
\usepackage{graphicx}
% \usepackage{biblatex}
% \addbibresource{bibliography.bib}

\title{Blending SIR and Predator-Prey Models to Predict the Labor Market}

\author{Jason Vasquez \and Dylan Skinner \and Benjamin Mcmullin \and Ethan Crawford}

\date{December 5 2023}

\begin{document}

\begin{abstract}
    
    The labor market, including the unemployment rate and the amount of workers looking for jobs, can have a large impact on the economny.
    The more people employed means more money being spent, which in turn means more money being made. 
    Furthermore, rise in unemployment can lead to a recession. Being able to predict the labor market can help us prepare for a recession and help us understand the economy better.
    In this paper, we adapt an SIR model to characterize the dynamics of employed, unemployed, and retired individuals in the labor market. 
    Additionally, we employ a quasi predator-prey model to illustrate the oscillatory behavior observed in the white-collar and blue-collar industries. 
    By comparing the SIR model to the predator-prey model, we aim to enhance our understanding of the complex interactions within the labor market, 
    providing potential insights for recession prediction and economic analysis.
    
\end{abstract}

\maketitle

We give permission for this work to be shared by ACME Volume 4 instructors and anybody else who may have reasonable motivation.

\section{Background/Motivation}

One thing that is certain in life is that people will always need jobs. Not only this, but people 
will often lose their jobs. Furthermore, people will (eventually) retire from their jobs.
The focus of our project is modeling this situation.

Our investigation navigates the intricate dynamics of employment trajectories and occupational sectors. 
Employing the SIR (Susceptible-Infectious-Recovered) model, typically used for disease dynamics but adapted 
here to study employment dynamics, we aim to comprehend the propagation of employment statuses—specifically, 
the transitions between being employed, unemployed, and retired. A discernible trend has emerged in recent times, 
notably influenced by the technological revolution. The surge in interest and demand for tech-oriented careers 
has prompted a significant shift away from traditional blue-collar professions. This migration has led to a 
dual challenge: a scarcity of skilled workers in the blue-collar sector and an oversaturation of the tech 
industry. Our study extends beyond the conventional SIR model, incorporating elements of a quasi-predator-prey 
framework inspired by ecological models. This approach allows us to capture the nuanced relationship between 
the blue-collar and white-collar industries, offering insights into the cyclical dynamics between these sectors. 
Motivated by the imperative to comprehend and address the consequences of this evolving employment landscape, 
our research aims to contribute valuable insights for informing strategic policies and industry interventions.

\section{Modeling}

\subsection{Theoretical Framework}

The Susceptible-Infectious-Recovered (SIR) model, developed by Kermack and McKendrick in 1927, 
is a foundational mathematical framework for understanding the spread of infectious diseases in populations. 
It divides individuals into susceptible, infectious, and recovered compartments, capturing the dynamics of disease transmission.
In our model, we adapt the SIR model to represent the dynamics of the employment market through the labor force,
unemployed, and retired populations.

\subsection{Previous Work}

We begin by building off the work of ElFadily et. al.~\cite{ElFadily}. In their work, ElFadily et. al. proposed a model
 representing the labor force and unemployed populations. They begin by defining their equations as

\begin{align}
    \begin{split}
        \frac{dL}{dt} &= \gamma U - (\sigma + \mu)L, \\
        \frac{dU}{dt} &= \rho \left(1 - \frac{L_{\tau} + U_{\tau}}{N_c} \right)L_{\tau} + \sigma L - (\gamma + \mu)U, \\
    \end{split}
\end{align}

where $L$ is the labor force, $U$ is the unemployed population, with initial conditions for (1) defined as:

\begin{align}
    \begin{split}
        L&(0) > 0, \quad U(0) > 0, \\
        (L(\theta),U(\theta)) &= (\varphi_1(\theta), \varphi_2(\theta)), \quad \theta \in [-\tau,0], \\
    \end{split}
\end{align}
where $\varphi_i\in C([-\tau, 0], \mathbb{R}^+),\;\; i=1,2$.

The parameters are defined as follows:

\begin{itemize}
    \item $\gamma$: employment rate
    \item $\sigma$: rate of job loss 
    \item $\mu$: mortality rate
    \item $\rho$: maximum population growth rate 
    \item $N_c$: population carrying capacity 
    \item $\tau$: time lag needed to contribute in the reproductive process of a new individual looking for a job
\end{itemize}

\subsection{Modifications: The Retirement Group}
 

With this information in mind, we can begin to adapt this model to fit our needs. We begin by adding a third population, the retired population, $R$.
We can then define our new equations as

\begin{align}
    \begin{split}
        \frac{dL}{dt} &= \gamma U - (\sigma + \mu)L {\color{red}\;- \left(\frac{\Sigma}{L + U}\right) L + \omega\left(\frac{\Sigma}{L + U}\right) R + \rho R},\\
        \frac{dU}{dt} &= \rho\left(1 - \frac{L + U}{N_c} \right)L + \sigma L -(\mu + \gamma)U,  \\
        \frac{dR}{dt} &= {\color{red}\left(\frac{\Sigma}{L + U}\right) L - \omega\left(\frac{\Sigma}{L + U}\right) R \; - \, \mu R},
    \end{split}
\end{align}

which simplify to 

\begin{align}
    \begin{split}
        \frac{dL}{dt} &= \gamma U - (\sigma + \mu)L {\color{red}\; + \;(\omega R - L)\left(\frac{\Sigma}{L + U}\right) + \rho R},\\
        \frac{dU}{dt} &= \rho\left(1 - \frac{L + U}{N_c} \right)L_{\tau} + \sigma L -(\mu + \gamma)U  \\
        \frac{dR}{dt} &= {\color{red}(L - \omega R)\left(\frac{\Sigma}{L + U}\right) \; - \, \mu R}.
    \end{split}
\end{align}

One of the first things to note from our equations is the removal of the time lag $\tau$. 
This is because, instead of factoring in people when they are born, we are instead factoring them
in when they turn of working age (16). This reduces unnecessary complexity in our model. Another thing to note
is that we have added two new parameters, $\omega$ and $\Sigma$. We define $\Sigma$ to be the number of people who
retire each year in the United States, and $\omega$ to be the rate at which retired people enter back
into the full-time workforce (which is a dimensionless constant). We can then define our new initial conditions as:

\begin{align}
    \begin{split}
        L(0) > 0, \quad &U(0) > 0, \quad R(0) > 0, \\
        (L(\theta),U(\theta), R(\theta)) &= (\varphi_1(\theta), \varphi_2(\theta), \varphi_3(\theta)), \quad \theta \in [-\tau,0],
    \end{split}
\end{align}
where $\varphi_i\in C([-\tau, 0], \mathbb{R}^+),\;\; i=1,2,3$. Incorporating nuanced dynamics into our model, we introduce the following terms and elucidate their significance within the equations:

% \begin{itemize}
%     \item $\pm\left(\frac{\Sigma}{L + U}\right) L$: This term captures the dynamic interplay of individuals retiring from the workforce each time step. By leveraging $\Sigma$, the annual number of retirees, and $L + U$, the total workforce population, we calculate the percentage of individuals retiring at each time step. Multiplying this by $L$, the number of individuals actively engaged in the workforce, yields the precise count of those retiring, excluding unemployed individuals due to their non-participation in this realistic scenario.

%     \item $\pm \omega\left(\frac{\Sigma}{L + U}\right) R$: This term accounts for individuals retiring and subsequently re-entering the workforce. The dimensionless constant $\omega$ denotes the rate at which retired individuals seamlessly transition back to full-time work. By multiplying $\omega R$—the expected number of retired individuals returning to the workforce—by $\frac{\Sigma}{L + U}$, we ascertain the percentage of recent retirees contributing to the workforce each time step.
    
%     \item $\rho R$: This term represents the natural growth of jobs, proportional to the number of employed people. If the number of employed people is increasing, we assume the economy is doing well, so there will be new jobs created.

%     \item $-\mu R$: This term represents the natural attrition of retired individuals through mortality each time step. Its presence acknowledges the inevitable and inherent departure of individuals from the system due to mortality, contributing to a comprehensive understanding of the model's temporal evolution.
% \end{itemize}

\begin{itemize}
    \item $\pm\left(\frac{\Sigma}{L + U}\right) L$: Captures the dynamic interplay of individuals retiring from the workforce at each time step. The term considers the annual number of retirees ($\Sigma$) in relation to the total workforce population ($L + U$), calculating the percentage of individuals retiring and multiplying by the number of individuals actively engaged in the workforce ($L$), excluding unemployed individuals.

    \item $\pm \omega\left(\frac{\Sigma}{L + U}\right) R$: Accounts for individuals retiring and re-entering the workforce. The dimensionless constant $\omega$ denotes the rate at which retired individuals transition back to full-time work. The term calculates the expected number of retired individuals returning to the workforce.

    \item $\rho R$: Represents the natural growth of jobs, proportional to the number of employed people. Assumes economic well-being and the creation of new jobs.

    \item $-\mu R$: Represents the natural attrition of retired individuals through mortality at each time step. Acknowledges the inevitable departure of individuals due to mortality, contributing to a comprehensive understanding of the model's temporal evolution.
\end{itemize}


\subsection{Modifications: The White-Collar and Blue-Collar Groups}

We can further model the labor market by examining the relationship between two industries, commonly referred to as the white-collar and blue-collar industries.
In recent years, we have seen an explosion of jobs and interest in the white-collar field, specifically in the tech industry, while the blue-collar industry has seen a decline in interest.
This has led to an oversaturation of the white-collar industry and a scarcity of skilled workers in the blue-collar industry, which in turn has slowed growth in the white-collar industry and led to increased growth in the blue-collar industry.
This cyclical relationship mirrors that of a predator-prey relationship, and we can model it as such.

The classical predator-prey model is given by the following equations:

\begin{align}
    \begin{split}
        \frac{dx}{dt} &= \rho x - a x y, \\
        \frac{dy}{dt} &= -\mu y + \varepsilon a x y,
    \end{split}
\end{align}

where $x$ is the prey population, $y$ is the predator population, and $\rho$, $a$, $\mu$, and $\varepsilon$ are parameters. 
We can adapt this model to fit our needs by defining the following:

\begin{align}
    \begin{split}
        \frac{dx}{dt} &= \rho x \left(1-\frac{x}{k}\right) - a x y, \\
        \frac{dy}{dt} &= -\mu y + \varepsilon a x y + \beta y \left(1-\frac{y}{C}\right),
    \end{split}
\end{align}

where 

\begin{itemize}
    \item $x$: blue-collar population
    \item $y$: white-collar population
    \item $\rho$: Growth rate of blue-collar jobs
    \item $a$: Rate at which people switch from blue collar to white collar jobs
    \item $\mu$: Decay rate of white-collar jobs
    \item $\varepsilon$: Efficiency of white-collar jobs in utilizing blue-collar jobs
    \item $k$: Carrying capacity for blue-collar jobs
    \item $C$: Carrying capacity for white-collar jobs.
    \item $\beta$: Growth rate of white-collar jobs
\end{itemize}

Thus, we can interpret the new additions to our model as:

\begin{itemize}
    \item $\rho x (1-\frac{x}{k})$: This term adds a carrying capacity to the ``prey'' population (or blue collar population). This is because, while the blue-collar industry is growing, it is not growing at an exponential rate. Instead, it is growing at a rate that is proportional to the number of blue-collar workers.
    \item $\beta y (1-\frac{y}{C})$: This term adds a carrying capacity to the ``predator'' population (or white collar population). This is because the white-collar industry, unlike a traditional predator, cannot grow indefinitely. Instead, it is limited by the number of people in the workforce. Furthermore, the white-collar industry can grow independent of the blue-collar industry, which is also a modification from the traditional predator population.
\end{itemize}


\section{Results}

\subsection{Labor Force, Unemployement, and Retirement Analysis}

We now give the specific values for hyperparameters for our model and how we came to those values. We define the following:

\begin{itemize}
    \item $\sigma = 0.013905$: We derived this value by analyzing comprehensive data on total layoffs and discharges in the United States from 2000 to 2023, as documented by the Federal Reserve Economic Data (FRED)~\cite{FRED}. The calculated average resulted in $\sigma = 0.013905$.
    \item $\rho = 0.014577$: The maximum growth rate, denoted as $\rho$, was determined by examining an Excel spreadsheet provided by MacroTrends~\cite{MacroTrends}. After averaging the growth rates from 2000 to 2022, we incorporated three standard deviations to arrive at $\rho = 0.014577$.
    \item $\gamma = 0.6062$: The average employment rate from 2000 to 2022, obtained from the Bureau of Labor Statistics~\cite{BLS}, led to the computation of $\gamma = 0.6062$.
    \item $\mu = 0.008498$: Derived from an analysis of mortality data from 2000 to 2022~\cite{usafacts}, the mortality rate was calculated by taking the average number of deaths per 100,000 people. This resulted in $\mu = 0.008498$.
    \item $N_c = 260,000,000$: The population of individuals aged 18 and above in the United States in 2022 was calculated to be $260,000,000$, as reported by Kids Count~\cite{kidscount}.
    \item $\Sigma = 775,045$: The annual number of retirees in the United States, determined by examining Social Security Administration data from 2000 to 2021~\cite{ssa}, was calculated using the formula:
    \[
        \Sigma = \frac{1}{2021 - 2001}\sum_{i=2001}^{2021}(x_i - x_{i-1})
    \]
    where $x_i$ is the number of retirees in year $i$. This resulted in $\Sigma = 775,045$.
    \item $\omega = 0.063$: The rate at which retired individuals re-enter the full-time workforce, denoted as $\omega$, was identified as 6.3\% or $0.063$ based on research by Maestas \cite{maestas}, resulting in $\omega = 0.063$.
\end{itemize}

We began testing our model by running it for 60 years with the current numbers for the United States (see figure~\ref{fig:results_lur_1}). As you can see,
the model very quickly reaches what appears to be an equilibrium. Additionally, we can see the total population grow as time goes on, reflecting the expected population growth of the United States.

\begin{figure}[h]
    \centering
    \includegraphics[width=\textwidth]{figures/results_lur_1.png}
    \caption{Initial conditions: $L(0) = 157,000,000$, $U(0) = 6,500,000$, $R(0) = 48,590,000$.}
    \label{fig:results_lur_1}
\end{figure}

To test the robustness of our model, we ran it with different initial conditions that do not represent the current situation in the United States. 
We first decreased the number in the labor force, increased the number of unemployed, and decreased the number of retired. We made these changes rather conservative
by only slightly perturbing the real numbers. We then ran the model for 60 years (see figure~\ref{fig:results_lur_2}).
Parallel to figure 1, we can see that the model still reaches an equilibrium, despite the initial conditions being skewed from their true values.

\begin{figure}[h]
    \centering
    \includegraphics[width=\textwidth]{figures/results_lur_2.png}
    \caption{Initial conditions: $L(0) = 100,000,000$, $U(0) = 50,000,000$, $R(0) = 10,000,000$.}
    \label{fig:results_lur_2}
\end{figure}

\newpage


We ran our model, once again, against a different set of initial conditions. This time, we significantly decreased the number of people in the occupied
labor force, significantly increased the number of unemployed (ensuring that the number of unemployed was much greater than the number of people in the labor force), and moderately decreased the number of retired. We then ran the model for 60 years (see figure~\ref{fig:results_lur_3}).
As you can see, the model still reaches an equilibrium, despite the initial conditions being heavily skewed from their true values.

\begin{figure}[h]
    \centering
    \includegraphics[width=\textwidth]{figures/results_lur_3.png}
    \caption{Initial conditions: $L(0) = 18,000,000$, $U(0) = 200,000,000$, $R(0) = 10,000,000$.}
    \label{fig:results_lur_3}
\end{figure}

We ran a final test, this time having the number of retired people set as greater than the number of
people in the labor force and unemployed combined. We then ran the model for 60 years (see figure~\ref{fig:results_lur_4}).

\begin{figure}[h]
    \centering
    \includegraphics[width=.8\textwidth]{figures/results_lur_4.png}
    \caption{Initial conditions: $L(0) = 17,500,000$, $U(0) = 20,400,000$, $R(0) = 195,000,000$.}
    \label{fig:results_lur_4}
\end{figure}

Unlike the previous graphs, we can see that the model does not reach an equilibrium in 60 years. 
While the number of people in the labor force rises and the number of retired people falls, 
this chance does not appear to be significant enough to reach an equilibrium. However,
when ran again for $T = 100$ years, we can see that the model gets closer to an equilibrium, but 
still does not reach one. But, when ran for $T = 400$ years, we can see that the model 
does appear to reach an equilibrium (see figure~\ref{fig:results_lur_5}).

\begin{figure}[h]
    \centering
    \includegraphics[width=.8\textwidth]{figures/results_lur_5.png}
    \caption{While there are two sets of graphs here, they use the same initial conditions.
            The only difference is how long the model was ran for. The top graph was ran for $T = 100$ years, and 
            the bottom graph was ran for $T = 400$ years.
            Initial conditions: $L(0) = 17,500,000$, $U(0) = 20,400,000$, $R(0) = 195,000,000$.}
    \label{fig:results_lur_5}
\end{figure}


\subsection{White-Collar and Blue-Collar Analysis}

For our white- and blue-collar model, we experimented with different hyperparameters to see how they affected the model. 

In our first run of the model, we used parameters $\rho = 7$, $a = 5$, $\mu = 1$, $\varepsilon = .2$, $k = 3$, $\beta = 1$, $C = 1.5$. As we see, the model oscilates
slightly in the beginning, and then settles into a stable equilibrium (see figure~\ref{fig:results_wb_1}). The initial conditions come from data on the US Labor marekt and percentage of workers in white-collar or blue-collar jobs~\cite{BLS}.

\begin{figure}[h]
    \centering
    \includegraphics[width=.8\textwidth]{figures/blue_vs_white2.png}
    \caption{With the parameters $\rho = 7$, $a = 5$, $\mu = 1$, $\varepsilon = .2$, $k = 3$, $\beta = 1$, $C = 1.5$, we can see that the model oscilates
    slightly in the beginning, and then settles into a stable equilibrium. Zooming in between years $10-50$, we can see the oscillations more clearly.}
    \label{fig:results_wb_1}
\end{figure}

% \begin{figure}[h]
%     \centering
%     \includegraphics[width=.5\textwidth]{figures/blue_vs_white_zoomed.png}
%     \caption{Highlighting the oscillations between years $10$ and $50$.}
%     \label{fig:results_wb_2}
% \end{figure}

While our model does settle into an equilibrium, we find it important to note that our model is not very stable.

Consider two different sets of initial conditions, namely $rho = 7$, $a = 5$, $\mu = 2$, $\varepsilon = .2$, $k = 3$, $\beta = 1$, $C = 1.5$ and 
$\rho = 7$, $a = 5$, $\mu = 1$, $\varepsilon = .2$, $k = 3$, $\beta = 1$, $C = 1.5$. Note that the only difference between these two initial conditions is
how $\mu$ differs by a $1$. Despite this, our model predicts completely different results (see figure~\ref{fig:results_wb_3}).

\begin{figure}[h]
    \centering
    \includegraphics[width=.8\textwidth]{figures/bad_paramenters.png}
    \caption{The left graph corresponds to $\mu = 2$, while the right graph corresponds to $\varepsilon = 1$, others parameters are kept the same.}
    \label{fig:results_wb_3}
\end{figure}

Thus, our model is very sensitive to perturbations of initial conditions and requires further improvements.

\section{Analysis/Conclusions}

Overall, our model of the labor market shows remarkable stability. 
We can see that, regardless of the initial conditions, the model reaches an equilibrium, 
with the number of employed, unemployed, and retired individuals remaining relatively constant.
When we used initial conditions that reflected the current numbers for the United States,
the model saw relatively little change as time went on (see figure~\ref{fig:results_lur_1}).
With initial conditions that represented a larger than average unemployed population, the model corrected itself and reached a
similar equilibrium as the previous model (see figure~\ref{fig:results_lur_2}).
Finally, when presented with initial populations that were flipped, the model still stabilized to the same equilibrium
(see figure~\ref{fig:results_lur_3}).

Despite the strengths of our model, there are some weaknesses present, too. One weakness is that
changing the initial conditions can cause the results to differ significantly between each other results during the first few years.
While it is true that the solutions end up reaching similar values as $T$ grows, those first few years of difference can pose a problem.

One additional problem, which is the most significant, is that this model only considers how the labor markets interact
with each other. One major factor in the labor market is how the economy is, and our model does not take that into account.
Thus, one improvement that can be made is finding a way to include present economic conditions.

While harder to see with the scale on the graph, the retired population does increase over time. 
This is because, as the population grows, more people retire and we see a slow but steady growth, as expected.

The white-collar and blue-collar model, while not as robust as the labor force model, still shows some interesting results. It is interesting to see how the relationships in the model
caused oscillations in the different populations. The oscialltions are small enough that they are not visible on the graph, but they are still present, and can mimic the overall 
labor force where a swing of thousands of jobs is noticed by the economy as a whole. A strength of this second model is exactly that, being able to see the oscillations while
keeping the oscillations to a scale that would be realistic in the real world.

However, the model is not without its weaknesses. The model is very sensitive to changes in the hyperparameters, and even a small change can cause the model to behave very differently. We see
in figure~\ref{fig:results_wb_3} that a change of $1$ in $\mu$ causes the populations to entirely flip. Additionally, changing $\varepsilon$ from $.2$ to $1$ causes 
the white-collar population to swing wildly and see booms and busts equating to almost 100 million jobs. That massive of a swing is not realstic and not very useful for real world application.

Overall, these models of the labor force show interesting results, especially the SIR model for modeling the stability of the labor force. The predator-prey model for the white-collar and blue-collar
industries is wildly unstable, but when given proper hyperparamers demonstrates an interesting relationship between the two industries.


% Rest of your document goes here

\newpage 
\bibliographystyle{plain}
\bibliography{bibliography}


\end{document}
